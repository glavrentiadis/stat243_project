\documentclass[]{article}
\usepackage{lmodern}
\usepackage{amssymb,amsmath}
\usepackage{ifxetex,ifluatex}
\usepackage{fixltx2e} % provides \textsubscript
\ifnum 0\ifxetex 1\fi\ifluatex 1\fi=0 % if pdftex
  \usepackage[T1]{fontenc}
  \usepackage[utf8]{inputenc}
\else % if luatex or xelatex
  \ifxetex
    \usepackage{mathspec}
  \else
    \usepackage{fontspec}
  \fi
  \defaultfontfeatures{Ligatures=TeX,Scale=MatchLowercase}
\fi
% use upquote if available, for straight quotes in verbatim environments
\IfFileExists{upquote.sty}{\usepackage{upquote}}{}
% use microtype if available
\IfFileExists{microtype.sty}{%
\usepackage{microtype}
\UseMicrotypeSet[protrusion]{basicmath} % disable protrusion for tt fonts
}{}
\usepackage[margin=1in]{geometry}
\usepackage{hyperref}
\hypersetup{unicode=true,
            pdftitle={Final Project 2018 - STAT243},
            pdfauthor={Bunker, JD; Bui, Anh; Lavrentiadis, Grigorios},
            pdfborder={0 0 0},
            breaklinks=true}
\urlstyle{same}  % don't use monospace font for urls
\usepackage{graphicx,grffile}
\makeatletter
\def\maxwidth{\ifdim\Gin@nat@width>\linewidth\linewidth\else\Gin@nat@width\fi}
\def\maxheight{\ifdim\Gin@nat@height>\textheight\textheight\else\Gin@nat@height\fi}
\makeatother
% Scale images if necessary, so that they will not overflow the page
% margins by default, and it is still possible to overwrite the defaults
% using explicit options in \includegraphics[width, height, ...]{}
\setkeys{Gin}{width=\maxwidth,height=\maxheight,keepaspectratio}
\IfFileExists{parskip.sty}{%
\usepackage{parskip}
}{% else
\setlength{\parindent}{0pt}
\setlength{\parskip}{6pt plus 2pt minus 1pt}
}
\setlength{\emergencystretch}{3em}  % prevent overfull lines
\providecommand{\tightlist}{%
  \setlength{\itemsep}{0pt}\setlength{\parskip}{0pt}}
\setcounter{secnumdepth}{0}
% Redefines (sub)paragraphs to behave more like sections
\ifx\paragraph\undefined\else
\let\oldparagraph\paragraph
\renewcommand{\paragraph}[1]{\oldparagraph{#1}\mbox{}}
\fi
\ifx\subparagraph\undefined\else
\let\oldsubparagraph\subparagraph
\renewcommand{\subparagraph}[1]{\oldsubparagraph{#1}\mbox{}}
\fi

%%% Use protect on footnotes to avoid problems with footnotes in titles
\let\rmarkdownfootnote\footnote%
\def\footnote{\protect\rmarkdownfootnote}

%%% Change title format to be more compact
\usepackage{titling}

% Create subtitle command for use in maketitle
\newcommand{\subtitle}[1]{
  \posttitle{
    \begin{center}\large#1\end{center}
    }
}

\setlength{\droptitle}{-2em}

  \title{Final Project 2018 - STAT243}
    \pretitle{\vspace{\droptitle}\centering\huge}
  \posttitle{\par}
    \author{Bunker, JD; Bui, Anh; Lavrentiadis, Grigorios}
    \preauthor{\centering\large\emph}
  \postauthor{\par}
      \predate{\centering\large\emph}
  \postdate{\par}
    \date{December 10, 2018}


\begin{document}
\maketitle

\section{I. Purpose}\label{i.-purpose}

This project aims to apply the adaptive rejection sampling method
described in Gilks et al. (1992). The main function \textit{ars()}
generates a sample from any univariate log-concave probability density
function. In nonadaptive rejection sampling, the envelope and squeezing
functions, \(g_l(x)\) and \(g_u(x)\), are determined in advance. Under
the adaptive rejection sampling framework, these functions are instead
updated iteratively as points are sampled.

\section{II. Contributions of
members}\label{ii.-contributions-of-members}

Each team member was primarily responsible for one of the three steps
mentioned in Gilks et al. (1992): (1) the initialization step, (2) the
sampling step, (3) the updating step. Summary of contributions:

\begin{itemize}

\item Bunker, JD: Primarily responsible for the initialization step where the . If the user does not provide pointswhich samples the starting points. The \textit{initial()} function . Other contributions include improving the computational efficiency of the \textit{Update_accept()} function, and testing the log-concavity of the given function.

\item Lavrentiadis, Grigorios: Primarily responsible for the sampling step, which generates the value $x^*$ from a piecewise exponential distribution, $s_k(x)$. Assemble the auxiliary functions into the comprehensive \textit{ars()} function.

\item Bui, Anh: Update_accept function to decide whether $x^*$ is accepted or rejected in the final sample.  Update the envelop and squeezing functions accordingly.  Write the report and the formal testing.

\end{itemize}

\section{III. Theoretical background for rejection
sampling}\label{iii.-theoretical-background-for-rejection-sampling}

Assume f(x) is an univariate log-concave probability density function.
We sample from g(x), which is a scaling of function f(x). Assuming that
we have the envelop function \(g_{u}(x)\) and the squeezing function
\(g_{l}(x)\) of g(x).

Let Y \textasciitilde{} Unif(0,\(g_{u}(x)\)) and where w
\textasciitilde{} Unif(0,1). We reject \(x*\) if

\[
Y < g(x*) 
\] \[
\frac{Y}{g_{u}(x)} < \frac{g(x)}{g_{u}(x)}
\] \[
w < \frac{g(x)}{g_{u}(x)} 
\]

The paper gives the algorithm of adaptive rejection sampling when
working with the log of g(x), \(g_{l}(x)\), and \(g_{u}\), which will be
discussed more in details in the auxiliary functions section.

\section{IV. Auxiliary functions}\label{iv.-auxiliary-functions}

\subsection{1. Initial function}\label{initial-function}

\subsection{2. Generate x* from piecewise exponential
probabilities}\label{generate-x-from-piecewise-exponential-probabilities}

The \texttt{SamplePieceExp} function draws samples \(x^*\) out of a
piece-wise exponential distribution using the inverse sampling approach.
Initially a \(Pinv\) sample is drawn from a \(0\) to \(1\) uniform
distribution that corresponds to the cumulative probability of the
random sample \(x^*\) To find the bin at which \(x^*\) belongs, \(Pinv\)
is compared with the cumulative probability of each bin. \(x^*\) belongs
to the bin whose cumulative probability (\(Pcum_i\)) is the smallest out
of all bins that have \(Pcum\) larger than \(Pinv\). \(x^*\) is
estimated by solving the following cumulative probability equation,
where \(z_0\) is the left bound of the distribution and \(P_j\) is the
probability of bin \(j\).

\[
P_{inv} = \int^{x^*}_{z_0} s(x) dx = \sum^i_{j=1}P_j + \int^{x^*}_{z_i} s(x) dx
\]

\(DP\) equals to the probability \(P(z_i > x > x^*)\) where \(z_i\) is
the lower bound of the bin \(i\) where \(x^*\) belongs

\[
DP = \int^{x^*}_{z_i} e^{h(x_j) + (x-x_j) h'(x_j)} dx 
\]

To simplify the equation we define: \(a = h(x_j)-x_j h'(x_j)\) and
\(b = h'(x_j)\) From this equation, \(x^*\) equals to:

\[
x^* = \frac{1}{b} log(DP~b~e^{-a} +  e^{b~z_i})
\]

\subsection{3. Update accept function}\label{update-accept-function}

\textbf{Algorithm}

Inputs: w \textasciitilde{} Unif(0,1) \newline
\[l_{k}(x^{*}) = log(g_{l}(x^{*}))\]
\[u_{k}(x^{*}) = log(g_{u}(x^{*}))\] \[h(x^{*}) = log(g(x^{*}))\]
\[s_{k}(x) = exp(u_{k}(x))/\left(\int_{D} u_{k}(x') \; dx'\right) = g_{u}(x)/\left(\int_{D} g_{u}(x') \; dx'\right)\]

The lower bound of h(x) is \(l_{k}(x)\), which connects the values of
function h on abscissaes. The function of \(l_{k}(x)\) between two
consecutive abscissaes \(x_{j}\) and \(x_{j+1}\) is
\[l_{k}(x) = \frac{(x_{j+1} - x)h(x_{j}) + (x - x_{j})h(x_{j+1})}{x_{j+1} - x_{j}}\]
Let X be the domain of abscissaes, H be the domain of the realized
function H at abscissaes, H\_prime be the domain of the realized first
derivative of function H at abscissaes, Z be the domain of intersection
of tangent lines at abscissaes.

\[h'(x) = \frac{dlog(g(x))}{dx} = \frac{g'(x)}{g(x)}\] The intersection
of the tangents at \(x_{j}\) and \(x_{j+1}\) is

\[z_j = \frac{h(x_{j+1})-h(x_{j}) - x_{j+1}h'(x_{j+1})+x_{j}h'(x_{j})}{h'(x_{j})-h'(x_{j+1})}\]
Then for x between \(z_{j-1}\) and \(z_{j}\)
\[u_{k}(x) = h(x_{j}) + (x-x_{j})h'(x_{j})\]

\textbf{Step 1}: If \(w < exp(l_{k}(x^{*}) - u_{k}(x^{*}))\) \newline
- Accept \(x^{*}\) when the condition is satisfied. Draw another
\(x^{*}\) from \(s_{k}(x)\) \newline
- Reject \(x^{*}\) when the condition is not satisfied.

\textbf{Step 2}: These two procedures can be done in parallel. \newline
- Evaluate \(h(x^{*}), h'(x^{*})\). Update
\(l_{k}(x), u_{k}(x), s_{k}(x)\). X includes \(x^{*}\) as an element.
\newline
- Accept \(x^{*}\) if \(w < exp(h(x^{*}) - u_{k}(x^{*}))\). Otherwise,
reject.

Since the h(x), \(l_{k}(x)\), \(u_{k}(x)\) can be generated from vectors
H, H\_prime, and Z, we improve the efficiency of the calculation by
efficiently updating H, H\_prime, and Z. We append the vectors
associated with the new abscissae x* and append to the existing vectors.

Multiple testing are generated based on values that \(x^*\) can take.
For example, if \(x^*\) is out of the domain of X,
\(l_{k}(x^{*})=-Inf\). If \(x^*\) is at the minimum value X{[}1{]} and
maximum value X{[}n{]} in the domain of X, \(l_{k}(x^{*})\) will take
the values on the lines connecting X{[}1{]} and X{[}2{]}, and connecting
X{[}n-1{]} and X{[}n{]} respectively. Vector Z is also generated for the
case when H(x) is a linear function of x (for example, the exponential
distribution)

\section{V. Testing}\label{v.-testing}

\subsection{1. Formal tests for ars
function}\label{formal-tests-for-ars-function}

The ars function passes the test for the following distributions in the
testing phase using the Kolmogorov-Smirnov Test: \newline
- Normal distribution with mean = 0 and standard deviation = 1 \newline
- Normal distribution with mean = 7 and standard deviation = 2 \newline
- Beta(1,3) distribution \newline
- Gamma(2,3) distribution \newline
- Exponential(5) distribution \newline

\subsection{2. Tests for auxiliary
functions}\label{tests-for-auxiliary-functions}

\subsubsection{a. Test CalcProbBin
function}\label{a.-test-calcprobbin-function}

The CalcProbBin function generates the cumulative probability based on
elements in vector Z (i.e.~the intersection of the tangent lines of
abscissaes).Based on vector Z under the N(0,1) density, we test whether
the calculated cumulative probabilities are (nearly) equal to the
cumulative probability using ``pnorm(Z)''.

\subsubsection{b. Test UpdateAccept
function}\label{b.-test-updateaccept-function}

The UpdateAccept function decides whether to accept \(x*\) or not based
on designed conditions. We test this function by checking that if \(x*\)
is included in X, \(x*\) is accepted and included in the x\_accept
vector, as well as there is no update, i.e.~H and H\_prime stay the
same.


\end{document}
